
\section{Généralités}

Le système est modélisé sous forme d'une machine à états-transitions. Les états sont caractérisés par des constantes respectant des axiomes et des variables respectant des conditions nommées "invariants". Les constantes caractérisent le "contexte" du système i.e. l'aspect statique. Les variables décrivent l'aspect dynamique. Les transitions sont désignées par le terme event. Les events sont des actions conditionnées ou non par une garde. Une garde est une condition qui autorisera ou non l'action du système. La transition du système est supposée instantanée et se produire dès que la garde est vraie. Il peut survenir des cas d'indéterminisme externe lorsque plusieurs gardes sont vraies simultanément. Si elles sont toutes fausses, le système est bloqué. Si, à tout instant, une garde est vraie, le système ne se termine jamais.

\paragraph{}
La démarche itérative proposée dans l'exercice consiste à écrire les exigences a partir des spécifications ; chaque exigence est identifiée par un label qui permet la traçabilité et le classement en catégories notamment fonctionnelle et environnementale. On part d'un modèle abstrait simple, formalisé dans l'outil rodin. Des preuves notamment d'invariance sont effectuées. Le plugin Atelier B génère des obligations de preuves dans un langage correspondant à la logique du premier ordre associée à la théorie des ensembles.Atelier B utilise notamment une base de règles d’inférence associée à un moteur contenant des heuristiques et un prouveur de type SAT.

\paragraph{} Le modèle formel est corrigé le cas échéant par ajout de gardes ou de conditions sur les gardes si on cherche à éviter les blocages. Les exigences sont complétées si nécessaire. Enfin par raffinement successifs, on passe d'une vision synoptique du système à une vue plus détaillée qui peut conduire jusqu'à la phase d'implémentation.

