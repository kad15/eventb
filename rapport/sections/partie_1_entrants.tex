\subsection{Menaces d'éventuels entrants}

Pour les intégrateurs, les barrières à l'entrée semblent davantage relever de la présence de mastodontes tels que FedEx, UPS, TNT ou DHL capables de pratiquer des prix bas dûs aux économies d'échelles et disposant d'une forte intégration internationale et locale avec des hubs et maillages routiers optimisés. 


En outre, si la possibilité de louer ses avions plutôt que de les acheter permet de créer des compagnies ariennes fussent-elles éphémères, la sur-capacité actuelle de l'offre favorise plutôt la sortie que l'entrée. Si la demande est destinée à augmenter, la capacité l'est également avec le nombre croissant de \textit{wide-body} en circulation. Les compagnies souhaitant expérimenter le modèle mixte devront ainsi justifier d'un solide réseau de routes et de hubs, ce qui explique aujourd'hui la prépondérance des majors dans ce marché.  

A priori, les nouveaux entrants éventuels se porteront plutôt sur des marchés de niche, à moins qu'ils ne surfent sur l'apport de nouvelles technologies telles que les drones, les avions sans pilotes, le tout numérique pour réduire le coût des formalités et les accélérer, le centraliser. Mais alors les firmes en place ne sont-elles pas les mieux placées pour tirer partie de ces avancées quitte à racheter ces start-up ?