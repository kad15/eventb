\section{Rôle de l'ATM dans la stratégie des acteurs du fret aérien}

%Single European Sky ATM Research (SESAR)

Du fait des spécificités propres au secteur du fret aérien, l'ATM joue un rôle important dans l'organisation et la stratégie des différents acteurs. En effet, nous allons voir que le fait de transporter des marchandises implique des problématiques parfois bien différentes de celles du transport de passagers.

\subsection{Un trafic réparti}

\subsubsection{Des vols ayant principalement lieu la nuit}

Comme le montre différents articles tels que \cite{popescu}, le fret aérien a, dès ses origines, été principalement organisé avec des vols ayant lieu la nuit. En effet, cela a plusieurs avantages du point de vue de l'organisation du trafic aérien. 

Tout d'abord, cette activité étant en grande partie organisée autour de grands hubs \cite{Walcott201764}, cela permet aux marchandises de transiter le jour vers ces centres névralgiques pour ensuite être redistribuées dans le monde entier dans des avions chargés en toute fin de journée. 

Ensuite, les marchandises n'étant pas soumises aux mêmes contraintes temporelles que les passagers, leur transport dans des vols de nuit permet de répartir le trafic tout au long d'une journée : les heures de pointe sont alors consacrées aux vols passagers et les créneaux disponibles la nuit aux vols cargo. Cela permet donc de ne pas rajouter de trafic lors de périodes de congestion puisque les marchandises peuvent, dans la plupart des cas, partir en plein milieu de la nuit.

Une organisation de nuit peut également permettre à des entreprises de transport de fret de réduire certains coûts et d'assouplir leur organisation. En effet, ces vols ayant lieu hors période de congestion, les redevances aéroportuaires et de contrôle aérien peuvent être moins élevées qu'en plein pic de trafic. De même, la demande étant moindre, l'allocation de créneaux s'en trouve grandement facilitée.

Enfin, cette organisation autour de vols de nuit permet à certains acteurs de diversifier leur activité. On peut citer en France l'exemple de ASL Airlines France (anciennement Europe Airpost) qui possède plusieurs B737-300 \textit{Quick-Change} lui permettant d'effectuer des vols passagers le jour et des vols de fret la nuit.\\

On voit donc que les différents acteurs du transport aérien de fret prennent en considération un certain nombre de problématiques liées à l'organisation générale du trafic aérien et décident alors d'organiser principalement leur activité la nuit. Nous allons cependant voir que d'autres problématiques ATM peuvent porter atteinte à ce mode de fonctionnement.

\subsubsection{Un trafic de nuit remis en cause}

Comme nous venons de le voir, les transporteurs aériens de fret utilisent essentiellement des créneaux de nuit pour leurs vols. Cependant, principalement pour des raisons de gêne des riverains, le trafic de nuit est remis en cause sur de nombreux aéroports. Par exemple, un couvre-feu est mis en place sur l'aéroport de Paris-Orly entre 23H30 et 6H00 depuis 1963 afin d'éviter le survol par les aéronefs des zones habitées en plein milieu de la nuit.

Il est à noter que sur certains aéroports, le trafic de nuit peut être malgré tout important du fait de la présence du hub de certaines compagnies de fret. C'est le cas notamment à Paris-Charles-de-Gaulle où FedEx et d'autres compagnies de fret ont leur hub. 

Les articles \cite{boutelet_2012} et \cite{garric_2012} montrent par exemple en quoi la fin des vols de nuit imposée à l'aéroport de Francfort peut porter atteinte au développement du trafic de fret aérien.

Dans ce cadre, les compagnies sont dans l'obligation de revoir leur stratégie vis-à-vis de l'organisation du trafic. Les départs des vols de fret doivent alors se faire plus tôt ou plus tard et, parfois, prendre part à la congestion à certaines périodes.

Cette réduction des possibilités de vols de nuit peut également pousser les compagnies vers une réduction de l'utilisation d'avions tout-cargo au profit de l'embarquement de fret sur les vols passagers. C'est en effet une tendence forte du secteur comme le montre \cite{RePEc:eee:jaitra:v:61:y:2017:i:c:p:34-40}. Même si cette tendance ne trouve pas forcément son origine dans la réduction des vols de nuit, cette nouvelle contrainte ne fait que favoriser ce changement de stratégie. Cela permet en effet d'optimiser l'utilisation des capacités de transport des aéronefs passagers en embarquant du fret qui aurait pu être transporté par un avion tout-cargo et donc de réduire le trafic des aéronefs entièrement dédiés au fret. L'article montre même que certaines compagnies, comme Air France, éliminent les avions tout-cargo de leur flotte au profit de ce mode de transport des marchandises.

Dans le cas où les aéroports continuent de permettre les vols de nuit, les problèmes de voisinage liés à ces vols ont d'autres impacts sur la gestion du trafic. Par exemple, de nombreux aéroports ont mis en place des procédures particulières pour que la gêne des riverains soit diminuée la nuit. Il s'agit dans la plupart des cas de nouvelles procédures aux instruments plus précises, évitant le survol de certaines zone voire obligeant les aéronefs à survoler un endroit précis au-dessus d'une certaine altitude. Ces procédures mises en place montrent bien le rôle que peut avoir l'ATM pour permettre aux compagnies de fret de continuer à voler la nuit sur certains aéroports.\\

On voit donc que des contraintes liées à la gestion du trafic de nuit peuvent imposer aux compagnies de fret de se réorganiser et de redéfinir en quelque sorte leur stratégie.

\subsection{Un réseau fortement maillé}

Comme le montre \cite{O'Kelly20141}, des compagnies de transport de fret comme FedEx bénéficient de plusieurs hubs dans le monde entier et d'un réseau de lignes très dense. Cela permet donc une optimisation très fine  du trafic et des coûts au sein de chaque compagnie. Voyons les implications que cela peut avoir du point de vue de l'ATM.

\subsubsection{Un "routage" facilité des marchandises}

La densité du réseau auquel ont accès les compagnies de fret permet du multiplier les chemins permettant, à partir d'un point A, d'arriver à un point B. En effet, une marchandise de FedEx partant de France et dont la destination finale est le centre des États-Unis pourra emprunter sans grande différence en terme de distance le hub de Memphis, d'Oakland ou d'Indianapolis.

Cela permet aux compagnies d'optimiser les différents chargements de leurs avions et, par là, de répartir sur différentes lignes les marchandises transportées, mais aussi de limiter le phénomène de retour à vide. Encore une fois, cela peut avoir des implications non négligeables en terme de gestion du trafic puisqu'au lieu de faire partir plusieurs avions vers une même destination, certaines charges transportées peuvent être réparties sur d'autres lignes déjà existantes.

Cette optimisation du "routage" des marchandises est, de plus, renforcée par le point qui suit.

\subsubsection{Un trafic très résilient face aux retards}

Comme nous avons déjà pu le remarquer, le transport de marchandises a pour spécificité de ne pas avoir les mêmes contraintes temporelles que les passagers.

En effet, lorsqu'un avion transportant des passagers est en retard ou annulé, un grand nombre de mesures doivent être mises en place pour la prise en charge de ces derniers et leur acheminement dans les meilleurs délais.

En ce qui concerne la plupart des marchandises, un retard de quelques heures n'a que très peu d'implications (sauf marchandises bien spécifiques comme dans le cas du transport d'animaux vivants). Ainsi, il est possible de faire le choix de routes rallongeant quelque peu le transport d'une marchandise sans que cela ait un réel impact sur le service rendu. Cela permet encore une fois d'optimiser la répartition des marchandises sur différentes lignes.

De même, les compagnies de fret étant, dans certains cas, très résilientes face aux retards, il leur est possible d'accepter, par exemple, des changements de créneau afin de désengorger le trafic sur un aéroport congestionné.\\

On voit donc en quoi la spécificité du transport aérien de fret a des implications sur la gestion générale du trafic aérien. Nous pouvons d'ailleurs également voir qu'un certain nombre de nouveaux développements dans l'ATM peuvent encore modifier les stratégies des acteurs du fret aérien.