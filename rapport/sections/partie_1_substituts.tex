\subsection{Menaces liées à l'existence de substituts}
\label{substituts}

Par son coût élevé, le transport de fret aérien se justifie dans le cadre de certains produits typiques définis au § \ref{produits} pour lesquels il n'existe généralement pas de substituts satisfaisants.

Cependant, dans certains cas, l'optimisation et l'anticipation logistique des entreprises peuvent leur permettre d'être moins dépendantes du transport aérien. Avec des progrès techniques, en termes de conservation et de chaîne du froid par exemple, le transport maritime peut aussi parfois devenir une option viable. Le transport ferroviaire, que l'on estime 80 \% moins cher que le transport aérien, continue lui aussi de se développer notamment entre l'Europe et l'Asie, concurrençant sérieusement le transport maritime et grignotant peut-être demain le marché du fret aérien. \cite{lemonde_train}

Enfin, l'augmentation des coûts de production à l'étranger, le développement du protectionnisme avec les crises, les problématiques liées au respect de la propriété intellectuelle peuvent constituer autant de raisons pour les entreprise de relocaliser leur production. En ce sens, le transport terrestre redevient un concurrent sérieux. 
