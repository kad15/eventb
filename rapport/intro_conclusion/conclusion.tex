\section*{Conclusion}
\addcontentsline{toc}{section}{Conclusion}


Ce BE aura été l'occasion de se familiariser avec la méthode formelle event-b qui par modélisation d'un système sous la forme d'une machine à états-transitions propose une approche amont de la conception et vérification des systèmes, reposant sur un formalisme mathématique rigoureux, par opposition à une conception avale contrainte de s'appuyer sur des batteries de tests a posteriori pour "démontrer" empiriquement la correction du système.

\paragraph{}
La gestion de la complexité  dans cette méthode repose sur une raffinage successif du modèle par héritage et extension des attributs et des propriétés entre machines successives validées progressivement sur la base de règles d'inférences. Elle permet également d'approcher l'exhaustivité et l'exactitude des exigences en les complétant.
