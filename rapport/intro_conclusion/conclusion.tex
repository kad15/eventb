\section*{Conclusion}
\addcontentsline{toc}{section}{Conclusion}

%Cette étude réalisée par raffinement successif montre qu’un système complexe peut être modélisé et validé de manière formelle grâce à des outils relativement simple. La difficulté dans l’implémentation se situe alors plutôt dans l’interprétation des exigences et dans la validation du modèle. Il est de plus important de montrer qu’un système entièrement modélisé et validé n’impose pas forcément un système totalement fonctionnel. En effet notre seconde machine était validée même avant d’avoir ajouté notre condition de non blocage. Il est alors très important de se poser des questions sur la viabilité du modèle réalisé et de bien vérifié que notre système ne peut se bloquer sur des événements limités.
%Finalement ce projet nous a montré l’importance de réaliser notre modélisation de manière successive, et donc par raffinement afin de pouvoir valider petit à petit notre modélisation sans être perdu dans quelque chose de trop complexe. La validation d’un modèle simple permet d’arriver plus confiant sur un modèle de plus haut niveau.
\paragraph{}
