\documentclass[a4paper,12pt]{article}


% % % % % % % % % % % % % % % % % % % % % % % % % % % % %
% % ATTENTION : dans les figures le label doit être mis 
%APRES le caption pour que le numéro de la figure lorqu'on
%la référence soit le bon ; sinon on a le numéro du paragraphe
% % % % % % % % % % % % % % % % % % % % % % % % % % % % %

% package qui fournit \justify 
\usepackage[document]{ragged2e}

%pifont pour les puces de formes spéciales
\usepackage{pifont}

% césure exemple
% \hyphenation{an-ti-cons-ti-tu-tion-nel

\usepackage[utf8x]{inputenc}
\usepackage[T1]{fontenc}
\usepackage[frenchb]{babel} % If you write in French
%\usepackage[english]{babel} % If you write in English
\usepackage{lmodern} % Pour changer le pack de police
\renewcommand*\familydefault{\sfdefault}
\usepackage{makeidx}
\usepackage{amsthm}
\usepackage{amsmath}
\usepackage{amssymb}
\usepackage{mathrsfs}
\usepackage{stmaryrd}
\usepackage{geometry}
%\usepackage{graphicx}
\usepackage{graphbox}
\usepackage{supertabular}
\usepackage{tabularx}
\usepackage{longtable}
\usepackage{pdflscape}
\geometry{hmargin=2cm,vmargin=2cm}

\usepackage{booktabs}
\usepackage{tabularx}
\usepackage[table]{xcolor}
\usepackage{ltablex}
\usepackage{float}
\usepackage{url}

\usepackage{chngcntr}
\counterwithin*{footnote}{page}


\usepackage[titletoc,toc,title,page]{appendix}
\renewcommand{\appendixtocname}{Annexes}
\renewcommand{\appendixpagename}{Annexes}

\usepackage{standalone}
\usepackage{ifthen}
\usepackage{xstring}
\usepackage{calc}
\usepackage{pgfopts}
\usepackage{tikz}
\usetikzlibrary{positioning,shapes,shadows,arrows}

\usepackage{algpseudocode}
\usepackage{algorithm}
\makeatletter
\renewcommand{\ALG@name}{Algorithme}
\renewcommand{\listalgorithmname}{Table des algorithmes}

\newtheorem{theo}{Définition}[section]
\usepackage{mathtools, bm}
\usepackage{amssymb, bm}

\usepackage{hyperref}
\hypersetup{
    colorlinks=true,       % false: boxed links; true: colored links
    linkcolor=black,       % color of internal links
    citecolor=purple,       % color of links to bibliography
    urlcolor=blue          % color of external links
}

\usepackage{listings}

\definecolor{dkgreen}{rgb}{0,.6,0}
\definecolor{dkblue}{rgb}{0,0,.6}
\definecolor{dkyellow}{cmyk}{0,0,.8,.3}

\lstset{
  language        = php,
  basicstyle      = \small\ttfamily,
  keywordstyle    = \color{dkblue},
  stringstyle     = \color{red},
  identifierstyle = \color{dkgreen},
  commentstyle    = \color{gray},
  emph            =[1]{php},
  emphstyle       =[1]\color{black},
  emph            =[2]{if,and,or,else},
  emphstyle       =[2]\color{dkyellow}}



\usepackage{blindtext}
\usepackage{enumitem} % pour changer les puces dans \itemize


\date{\today}

\makeindex
\def\siecle#1{\textsc{\romannumeral #1}\textsuperscript{e}}
\newcommand{\argmax}{\mathop{\mathrm{argmax}}\nolimits}
\newcommand{\pgcd}{\mathop{\mathrm{pgcd}}\nolimits}

\makeatletter
\renewcommand{\pod}[1]{\allowbreak\mathchoice
  {\if@display \mkern 18mu\else \mkern 8mu\fi (#1)}
  {\if@display \mkern 18mu\else \mkern 8mu\fi (#1)}
  {\mkern4mu(#1)}
  {\mkern4mu(#1)}
}

\usepackage{wallpaper}

\begin{document}
\renewcommand{\labelitemi}{\textbullet}
% pour factoriser l'échelle des figures 
%utilisation scale=\scaledvwa au lieu de scale = 0.3 ... 
\newcommand{\scaledvwa}{0.4} 
\newcommand{\scaledvw}{0.3}
\newcommand{\scalekad}{0.45}


\phantomsection
\begin{titlepage}
	\parindent=0pt
\ThisTileWallPaper{1.3\paperwidth}{1.0\paperheight}{images/ctrl2}
 
\addtolength{\wpXoffset}{-4.5cm}

%	\vspace*{\stretch{1}}
%	\begin{center}
%		\includegraphics[scale=0.5]{images/enac.png}%
%	\end{center}
	
\color{white}{	\vspace*{\stretch{1}} }
	\hrulefill
	\begin{center}\bfseries\Huge
		\color{white}
		{Ingénierie Système basée sur les modèles : Analyse fonctionnelle d'un système ATC simplifié de contrôle du trafic aérien} 
	\end{center}
	\hrulefill
	
	\vspace*{1cm}
	\begin{center}\bfseries\Large
			\color{white}
		{Jules HELLER - Abdelkader BELDJILALI}
		
	\end{center}
	
	\vspace*{\stretch{2}}


\end{titlepage}
%on créé la couverture

\pagebreak

\tableofcontents
\justify

\pagebreak

\section*{Introduction}
\addcontentsline{toc}{section}{Introduction}

Le transport aérien de fret s'est développé à partir de 1911 lors d'un
premier vol dont l'objet était de livrer 15 kg de courrier entre deux villes asiatiques distantes de 10 km sur un biplan Sommer. 

%Depuis, le traffic a "légèrement" augmenté pour atteindre en 2014, 51 millions de tonnes pour une valeur de US\$6.8 trillions \cite{RePEc:eee:jaitra:v:61:y:2017:i:c:p:34-40}.

Aujourd'hui, le fret aérien, au même titre que le transport de passagers, continue de jouer un rôle majeur dans l'économie, le développement international et plus fondamentalement dans notre manière d'appréhender le monde au même titre que son pendant virtuel, l'Internet. 


%La présente étude vise à effectuer une analyse de marché suivant le modèle des 5 forces de Porter.
%Le modèle de Porter consiste à effectuer l'analyse de marché suivant 5 domaines :
%la concurrence dans le secteur, deux menaces : les nouveaux entrants et les substituts et deux déterminants : pouvoir de marché des clients et fournisseurs.
%
%\begin{figure}[H]
%	\begin{center}
%		\includegraphics[scale=0.48]{images/porter/porter_1}
%		\caption{Modèle de Porter}
%		\label{inclusion}
%	\end{center}
%\end{figure}
%
%
%Suivront une étude du rôle de l'Air Traffic Management (ATM) dans la stratégie des acteurs du secteur et de son évolution future au regard de ces stratégies des acteurs du secteur. 
%   






\section{Expression textuelle des exigences}


%\subsection{}
%\paragraph{}


\begin{figure}[H]
	\begin{center}	
		\includegraphics[scale=0.85]{images/enac}
		\caption{Diagramme hiérarchique des fonctions}
		\label{hier}
	\end{center}
\end{figure}
\section{Expression textuelle des exigences}


%\subsection{}
%\paragraph{}


%\begin{figure}[H]
%	\begin{center}	
%		\includegraphics[scale=0.85]{images/enac}
%		\caption{Diagramme hiérarchique des fonctions}
%		\label{hier}
%	\end{center}
%\end{figure}


\subsection{Analyse des spécifications}

\paragraph{}

Le système à concevoir est un \textbf{contrôleur aérien automatique} chargé de contrôler le trafic sur le tarmac et d'autoriser l'accès à la piste des avions prêts au décollage. 
\paragraph{}
Ce système est un logiciel qu'on appelera le Contrôleur. Ce contrôleur interagit avec son environnement à savoir les pilotes et des capteurs non précisés dans l'énoncé. Les spécifications sont incomplètes car le contrôleur doit forcément autoriser ou refuser les avions à l'atterrissage également. Il est nécessaire d'avoir un capteur qui permet de compter les avions qui entrent et sortent du tarmac et un autre capteur qui assure le comptage des avions sur la piste. Les avions ne peuvent atterrir que si l'aéroport n'est pas plein, 20 places maximum sont disponibles. 
\paragraph{}
On suppose que les clearances sont données aux avions qui se trouvent sur le tarmac. Dès l'autorisation de décoller donnée, l'avion se rend sur la piste et décolle. Il n'y a pas d'étape intermédiaire. Un système de communication sécurisé de type data-link doit exister pour donner les clearances et recevoir les identifiants et les demandes des pilotes.

\paragraph{}
On suppose qu'il ne peut y avoir qu'un seul avion sur la piste ; pas de possibilité
de mettre deux avions au décollage en même temps.


\subsection{Reformulation des exigences}

\begin{table} [H]
	
	\centering

\begin{tabular}{| p{1.7cm} | p{15cm}|}

	\hline \rowcolor{yellow}
\Centering	\textbf{Label}& \Centering \textbf{Exigence}  \\ 

	\hline 
	FON-1&Le contrôleur doit autoriser les avions à décoller et atterrir  \\ 
	\hline 
FON-2	& Le nombre d'avions sur le tarmac est limité à 20 y compris ceux en attente  \\ 
	\hline 
	FON-3& Des avions entrent sur la piste et la quittent  \\ 
	\hline 
	FON-4& Des avions entrent sur le tarmac et le quittent  \\ 
	\hline 
	FON-5& La piste ne peut être occupée par plus de un avion \\ 
	\hline 
	FON-6& Le nombre de décollages ou atterrissages successifs n'est pas limité   \\ 
	\hline 
	FON-7& Le contrôleur doit fixer et délivrer les clearances à l'avance   \\ 
	\hline 
	FON-8& Le contrôleur ne doit autoriser l'avion qu'après l'envoi de son identifiant    \\ 
		\hline
	FON-9& Le contrôleur doit soit refuser, soit accepter, soit mettre en attente l'avion demandeur   \\ 
 	\hline
 FON-10& Le contrôleur doit refuser la clearance après 10 de mise en attente.   \\ 
	\hline 
   ENV-1 &Tout avion se dirigeant vers la piste doit avoir une autorisation de décoller \\ 
   	\hline 
   ENV-2 &Le système est muni d'un capteur qui permet de compter les avions sur la piste \\ 
   \hline 
   ENV-3 &Le système est muni d'un capteur qui permet de compter les avions sur la tarmac \\ 
	\hline 
\end{tabular} 
\caption{Tableau des exigences}
\end{table}



\section{Rôle de l'ATM dans le futur par rapport aux stratégies des acteurs }

L'\textit{Air Traffic Management} est un domaine en perpétuelle évolution, tout particulièrement ces dernières années en Europe. En effet, il s'agit d'un secteur où les procédures et les outils doivent évoluer et s'adapter au trafic afin d'assurer une meilleure gestion du trafic tout en gardant un niveau de sécurité élevé. Nous allons voir un certain nombre d'évolutions qui pourraient avoir un impact sur les acteurs du fret aérien.

\subsection{L'opportunité du programme SESAR}

Comme le montre \cite{52008DC0750}, le programme \textit{Single European Sky ATM Research} (SESAR) mis en place par la Commission Européenne affiche d'ambitieux objectifs tels que :
\begin{itemize}
\item une réduction de moitié des coûts de contrôle aérien;
\item une réduction de 10\% de l'impact sur l'environnement;
\item une division du risque d'accident par 10;
\item un triplement de la capacité de l'espace aérien.
\end{itemize}

Dans ce contexte, les entreprises de fret aérien, au même titre que l'ensemble des compagnies aériennes volant en Europe, vont voir apparaitre la possibilité d'augmenter leurs capacités et de diminuer leurs coûts. Ainsi, si les échanges commerciaux continuent à croitre, notamment avec l'Asie, ces entreprises de transport pourront se développer sans contraintes immédiates dues à la gestion du trafic aérien.

Ces objectifs pourront être rendus possibles par le progrès technique (moteurs plus économiques et moins bruyants), la généralisation de procédures efficaces (descente continue) et la refonte du ciel européen qui souffre aujourd'hui d'une coûteuse segmentation et d'un manque d'harmonisation et de standardisation.\\

On voit donc en quoi les travaux actuels de modernisation de l'ATM permettent d'ouvrir un certain nombre de perspectives aux entreprises de transport aérien en général et, par conséquent, aux entreprises de transport aérien de fret en particulier.

\subsection{Introduction de nouveaux paradigmes}

Dans le cadre des réflexions relatives à la privatisation de certains ANSP (\textit{Air Navigation Service Provider}), des idées de services commerciaux que pourraient vendre ces entités ont émergé.

Par exemple, il serait envisageable de moduler le tarif de la redevance de contrôle en fonction du gain de temps ou du retard que serait prête à accepter une compagnie. Ainsi, une compagnie qui souhaiterait être prioritaire sur l'attribution d'un créneau paierait une redevance plus élevée qu'une compagnie prête à accepter un retard.

Or nous avons pu voir que l'une des spécificités du transport aérien de fret est que les marchandises sont, dans le cas général, résilientes face aux retards. Ainsi, grâce à ces services commerciaux liés à la gestion du trafic, les transporteurs aériens de fret pourraient optimiser leurs coûts dans certains cas.\\

On voit donc que de nouveaux paradigmes émergent au sein des ANSP et que ceux-ci peuvent impacter fortement la stratégie des transporteurs aérien de fret.

\subsection{Vers une extension du périmètre du fret}

%Etablissement de régulation lors du développement des drônes et avions sans pilotes : voilure mobile ou fixe. 

Comme le montre \cite{RePEc:eee:jaitra:v:61:y:2017:i:c:p:34-40}, de nouvelles entrées, certes limitées mais réelles, font leur apparition sur le marché du transport aérien de fret. Il s'agit principalement d'opérateurs de drones souhaitant réaliser un transport de marchandise par le biais de cette nouvelle technologie.

On peut citer notamment des entreprises comme La Poste \cite{gradt_2016} ou Amazon \cite{figaro_2016} qui tentent de mettre en place ce nouveau marché.

Si ce nouveau segment répond plutôt à la problématique du dernier kilomètre qui ne concerne pas tous les acteurs du transport aérien de fret, il est à noter que les évolutions de l'ATM vont jouer un rôle essentiel dans le développement de ce marché.

En effet, l'automatisation du transport aérien de marchandise pose un grand nombre de difficultés au regard de la gestion du trafic aérien dans certains espaces. De nouveaux concepts émergent alors : on parle ainsi de l'UTM (\textit{Unmanned Aircraft Systems} (UAS) \textit{Traffic Management}) au lieu de l'ATM pour désigner ces problématiques de gestion de trafic propres aux drones.\\ 

Ainsi, les évolutions technologiques dans les domaines de l'ATM, de l'UTM et des drones apporteront de nouvelles possibilités de développement aux acteurs du transport aérien de marchandises.






\section*{Conclusion}
\addcontentsline{toc}{section}{Conclusion}

%Cette étude réalisée par raffinement successif montre qu’un système complexe peut être modélisé et validé de manière formelle grâce à des outils relativement simple. La difficulté dans l’implémentation se situe alors plutôt dans l’interprétation des exigences et dans la validation du modèle. Il est de plus important de montrer qu’un système entièrement modélisé et validé n’impose pas forcément un système totalement fonctionnel. En effet notre seconde machine était validée même avant d’avoir ajouté notre condition de non blocage. Il est alors très important de se poser des questions sur la viabilité du modèle réalisé et de bien vérifié que notre système ne peut se bloquer sur des événements limités.
%Finalement ce projet nous a montré l’importance de réaliser notre modélisation de manière successive, et donc par raffinement afin de pouvoir valider petit à petit notre modélisation sans être perdu dans quelque chose de trop complexe. La validation d’un modèle simple permet d’arriver plus confiant sur un modèle de plus haut niveau.
\paragraph{}



%\newpage
%\appendix
%\include{annexes/annexe_A}
%\section{Annexe}
\label{annexe_A}

\begin{itemize}
\item Pour écrire simplement les clés sur le disque, nous utilisons Marshal, qui garantit la compatibilité entre toutes les plateformes pour une même version de OCaml.
\item Bien que BatIO propose une API pour manipuler les canaux au niveau du bit, nous avons préféré rester au niveau de l'octet car il s'agit d'une solution plus évolutive – rares sont les bibliothèques proposant ce genre de fonctions. D'ailleurs, son fonctionnement est identique à ce que nous implémentons, reposant sur une lecture octet par octet.
\item Dans la version actuelle du code, les canaux d'entrées-sorties ne sont pas toujours fermés proprement lorsqu'une exception « fatale » est rencontrée…
\item Les blocs chiffrés sont écrits sur le canal de sortie sous forme de chaîne de caractères. Ainsi, le message chiffré constitué de deux blocs « 1234 5678 » est écrit, sous forme hexadécimale, «~31 32 33 34 00 35 36 37 38 00~». Pour déchiffrer, on lit donc le canal d'entrée « chaîne par chaîne ». Cela a l'avantage de produire une sortie lisible mais présente l'inconvénient de consommer bien plus d'espace qu'une représentation binaire qui serait spécialement conçue pour le problème.
\end{itemize}


\newpage
\nocite{*}  %affiche toutes les entrées du bib même celles qui ne sont pas citées.
% cf.    http://www.tuteurs.ens.fr/logiciels/latex/bibtex.html
% compilation en TROIS PHASE  bibtex traite un fichier *.aux mais bibtex mon_fichier comme bibtex mon_fichier.aux sont acceptés 
% latex mon_fichier.tex
% bibtex mon_fichier
% latex mon_fichier.tex


% \renewcommand{\bibname}{Toto}
% ou
%\renewcommand{\refname}{Bibliographie}
% dans le préambule.
%\bibliographystyle{alpha}
%\bibliography{references}
\input{page-de-couverture/page_blanche}
\pagebreak
\thispagestyle{empty}
\ThisTileWallPaper{1.45\paperwidth}{1.0\paperheight}{images/ctrl1}
\addtolength{\wpXoffset}{-4.5cm}


\justify








\end{document}
